\documentclass[utf8,english]{gradu3}

\usepackage{xcolor}

% NOTE: This must be the last \usepackage in the whole document!
\usepackage[bookmarksopen,bookmarksnumbered,linktocpage]{hyperref}

\addbibresource{thesis.bib} % The file name of your bibliography database

% My custom macros
\newcommand{\todo}[1]{\textbf{\textcolor{red}{#1}}}

% Actual document

\begin{document}

\title{Maintainability in cloud-native architecture}
\translatedtitle{Ylläpidettävyys pilvinatiiveissa arkkitehtuureissa}
\studyline{Software and Telecommunication Technology}
\avainsanat{%
  ylläpito,
  ylläpidettävyys
  julkinen pilvi,
  pilvinatiivi,
  pilviarkkitehti,
  arkkitehtuuri,
  ohjelmistoarkkitehtuuri,
  pro gradu -tutkielmat
}
\keywords{%
  maintenance,
  maintainability,
  public cloud,
  cloud-native,
  cloud architect,
  architecture,
  software architecture,
  Master's Theses}
\tiivistelma{%
  Tutkielman tavoitteena on selvittää kuinka ylläpidettävyys huomioidaan
  pilvinatiivien sovellusten arkkitehtuurisuunnitteluvaiheessa.  Tämän
  saavuttamiseksi suoritan kyselyn Nordcloud-yrityksen pilviarkkitehtien
  keskuudessa.  Ensin määritän kuinka tärkeänä ylläpidettävyyttä pidetään.
  Sitten kategorisoin ehdotukset ylläpidettävyyden huomioimiseen, ja suhteutan
  ne vastaajien kokemuksen määrään kohdealueelta. Lopuksi vertaan kyselyn
  tuloksia kirjallisuudessa ehdotettuihin lähestymistapoihin.
}
\abstract{%
  Goal of the thesis is to investigate how maintainability is addressed during
  the architectural design phase of cloud-native software development lifecycle.
  To this end, I conducted a survey among cloud architects of company Nordcloud,
  where I work.
  First I ascertain the perceived importance of maintainability.
  Then I categorize the suggested approaches for addressing maintainability and
  relate this to the respondents' years of experience in the target domain.
  Finally, I compare the survey results to approaches suggested in the literature.
}

\author{Juho Kettunen}
\contactinformation{\texttt{juho.kettunen@student.jyu.fi}}
\supervisor{Oleksiy Khriyenko}

\maketitle

\mainmatter

%%%%%%%%%%%%%%%%%%%%%%%%%%%%%%%%%%%%%%%%%%%%%%%%%%%%%%%%%%%%%%%%%%%%%%%%%%%%%%
\chapter{Introduction}

Goal of the thesis is to investigate how maintainability is addressed during
the architectural design phase of cloud-native software development lifecycle.
I chose this topic for three reasons:
\begin{enumerate}
  \item Choices made during the architectural design phase cascade into
        development and maintenance phases.
        Mistakes and oversights are slower and more expensive to correct later. \todo{<SOURCE?>}
  \item Maintenance phase of the software lifecycle is prevalent.
        It takes up the majority of total lifetime and costs. \todo{<SOURCE?>}
  \item Cloud-native approach helps reduce time-to-market and move costs from
        capital expenditure to operational expenditure. \todo{<SOURCE?>}
        Utilizing a public cloud platform allows the company to leverage a
        highly scalable, reliable and secure infrastructure and a wide variety
        of easily integratable services. \todo{<SOURCE?>}
\end{enumerate}

\todo{REWRITE:}
First I will ascertain the perceived importance of maintainability.
Then I will categorize the suggested approaches for addressing maintainability
and contrast this to the respondents' years of experience in the target domain.
Finally, I will compare the survey results to approaches suggested in the literature.


%%%%%%%%%%%%%%%%%%%%%%%%%%%%%%%%%%%%%%%%%%%%%%%%%%%%%%%%%%%%%%%%%%%%%%%%%%%%%%
\chapter{Background}

My main database for researching existing literature was the JYKDOK search engine for international articles \parencite{jykdok}.
I decided to search keywords in abstracts, instead of full-text or only in title.
In my opinion the full-text search would have been too lax, and title-only too strict.
With these search terms I received between 4 to 473 results per search, which was sufficient:
\begin{itemize}
  \item (Abstract:"cloud native" AND Abstract:maintainability)
  \item (Abstract:"cloud native" AND Abstract:architecture)
  \item (Abstract:"cloud native" AND Abstract:maintenance)
  \item (Abstract:architecture AND Abstract:maintainability AND Abstract:cloud)
  \item (Abstract:serverless AND Abstract:maintainability)
  \item (Abstract:serverless AND Abstract:operations)
  \item (Abstract:"cloud-native" AND Abstract:quality AND Abstract:attribute)
  \item (Abstract:software AND Abstract:"quality attribute" AND Abstract:maintainability)
\end{itemize}

To my chagrin, I was unable to find full-text of some articles with promising abstracts.
In case of one specific article \footnote{\todo{MENTION THE STUDY}}, I emailed the authors, and they kindly sent me a PDF of it.

I also employed an ad-hoc lightweight version of snowballing from bibliographies of
promising sources, e.g. \textcite{thesis-time-tracking} \todo{<NOT THIS SOURCE, others?>}
%%%%%%%%%%%%%%%%%%%%%%%%%%%%%%%%%%%%%%
\section{Cloud-nativity}

%%%%%%%%%%%%%%%%%%%%%%%%%%%%%%%%%%%%%%
\section{Architecture as a software lifecycle phase}

%%%%%%%%%%%%%%%%%%%%%%%%%%%%%%%%%%%%%%
\section{Maintainability}

%%%%%%%%%%%%%%%%%%%%%%%%%%%%%%%%%%%%%%
\section{Maintainability in cloud-native architecture}

%%%%%%%%%%%%%%%%%%%%%%%%%%%%%%%%%%%%%%%%%%%%%%%%%%%%%%%%%%%%%%%%%%%%%%%%%%%%%%
\chapter{Method}

%%%%%%%%%%%%%%%%%%%%%%%%%%%%%%%%%%%%%%%
\section{Research questions}

I aim to answer three research questions:
\begin{itemize}
  \item How much importance do cloud architects place on maintainability during
        the design phase?
  \item How to address maintainability concerns when architecting cloud-native
        applications?
  \item Do the recommendations from literature match the views of architects
        working in the field?
\end{itemize}


%%%%%%%%%%%%%%%%%%%%%%%%%%%%%%%%%%%%%%
\section{Survey}

I conducted a survey that offers real-life insights into the research questions.
Survey manifests as a Google Form.
It also includes a data collection disclaimer and an agreement checkbox to proceed.


\subsection{Questions}
First half of the survey gathers some demographic insights.
The latter half is dedicated to the actual subject matter.
These are the questions:
\begin{enumerate}
  \item Years of experience in cloud-native architecture?
  \item Years of experience in software architecture in general?
  \item Years of experience in IT in general?
  \item How do you prioritize these software quality attributes when designing cloud-native
        architecture?
  \item How do you address maintainability with platform- and technology choices?
  \item How do you address maintainability with application architecture?
\end{enumerate}

Questions 1 through 3 are single-selection questions with these ranges:
\begin{itemize}
  \item 0-2
  \item 2-5
  \item 5-10
  \item 10+
\end{itemize}

Question 4 offers a selection of software quality attributes:
\begin{itemize}
  \item Performance
  \item Reliability
  \item Maintainability
  \item Scalability
  \item Security
\end{itemize}

The respondent should set these quality attributes in a priority order.
Available priority choices are:
\begin{itemize}
  \item Most important
  \item More important
  \item Important
  \item Less important
  \item Least important
\end{itemize}

Questions 5 and 6 are open questions, where the respondent can describe their thoughts through text in any format they like.
I interpret the answers to find common categories from the suggestions.

\subsection{Target audience}
The survey is targeted at cloud architects of \emph{Nordcloud, an IBM company}, where I work.
I posted links to the survey on Nordcloud's internal Slack channels to reach
most of our cloud architects.
These channels were used:
\begin{itemize}
  \item \#tech-infra
  \item \#aws
  \item \#azure
  \item \#google-cloud
\end{itemize}

Some of these communities have overlapping audience, but the total reach is about 1600 people.
I assume our company to employ a few dozen cloud architects.
My goal was to get 10-20 answers in order to conduct any statistical analysis on
the data.
This goal was met within a week with 13 answers, with a final tally of \todo{<NUMBER>}.

\subsection{Timeline}
Survey was sent out on Monday 20.03.2023.
I was prepared to repost the survey in Slack up to 4 times once a week, until I
decide I have enough answers to proceed.
In the end there was no need to repost, as a sufficient number of architects
responded after the initial posting.
In early April I was able to convince one additional architect in-person at our
Jyväskylä office to respond.
Some stray responses were still recorded up to 21.4.2023.

\subsection{Data privacy concerns}
I use my JYU Google account for all Google-related functions.

All answers are collected and analysed anonymously.
Email addresses or other pieces of personal information are not collected.
The form requires logging in with a Google account.
This is enforced only to avoid accidental re-submissions.

Individual answers and their aggregated analysis will be stored in Google Forms and
Google Sheets in my student account until the end of my study rights and
subsequent removal of the account.


%%%%%%%%%%%%%%%%%%%%%%%%%%%%%%%%%%%%
\section{Analysis}

\begin{itemize}
  \item Calculate a relative grade for priority of maintainability based on answers to question 4.
  \item Interpret and list the suggestions from answers to questions 5 and 6.
  \item Normalize the suggestions to find common categories.
  \item Find the most popular categories and point out possible outliers.
  \item Compare categories to solutions proposed in literature.
  \item Bonus: Calculate statistical correlation between years of experience and:
        \begin{itemize}
          \item perceived importance of maintainability
          \item number of categories proposed in 5 and 6
          \item popularity of categories proposed in 5 and 6
        \end{itemize}
\end{itemize}


%%%%%%%%%%%%%%%%%%%%%%%%%%%%%%%%%%%%%%%%%%%%%%%%%%%%%%%%%%%%%%%%%%%%%%%%%%%%%%
\chapter{Results}

%%%%%%%%%%%%%%%%%%%%%%%%%%%%%%%%%%%%
\section{Demographics (Q1-Q3)}

%%%%%%%%%%%%%%%%%%%%%%%%%%%%%%%%%%%%
\section{Prioritization of maintainability (Q4)}

\begin{table}[!h]
  \begin{center}
    \caption{Priority choices and their numerical weights}
    \label{table:priorities1}
    \begin{tabular}{|c|c|}
      \hline
      \textbf{Priority} & \textbf{Weight} \\
      \hline
      Most important & 5 \\
      More important & 4 \\
      Important & 3 \\
      Less important & 2 \\
      Least important & 1 \\
      \hline
    \end{tabular}
  \end{center}
\end{table}


\begin{table}[!h]
  \begin{center}
    \caption{Average priorities and other metrics for the quality attributes (QA)}
    \label{table:priorities2}
    \begin{tabular}{|c|c|c|c|c|}
      \hline
      \textbf{QA} & \textbf{Average} & \textbf{Min} & \textbf{Median} & \textbf{Max} \\
      \hline
      Security & 4.07 & 1 & 5 & 5 \\
      Reliability & 3.80 & 1 & 4 & 5 \\
      Maintainability & 2.67 & 1 & 3 & 4 \\
      Scalability & 2.27 & 1 & 2 & 5 \\
      Performance & 2.20 & 1 & 2 & 4 \\
      \hline
    \end{tabular}
  \end{center}
\end{table}


In table \ref{table:priorities3} are additionally listed the distances between the average priorities.
From these values we can draw multiple conclusions.
The top two quality attributes, Security and Reliability, are unequivocally the most highly prioritized attributes in the minds of our cloud architects.
The distance between second and third most prioritized attributes is by far the largest individual jump in values.
Maintainability is sitting quite comfortable in the middle of the range.
This likely means it is considered important, but not a priority if sacrifices need to be made to software quality
Performance and scalability are considered least important items in an almost identical amount.

\begin{table}[!h]
  \begin{center}
    \caption{Average priorities and delta to next highest item}
    \label{table:priorities3}
    \begin{tabular}{|c|c|c|}
      \hline
      \textbf{QA} & \textbf{Average} & \textbf{$\Delta$} \\
      \hline
      Security & 4.07 & -\\
      Reliability & 3.80 & 0.27\\
      Maintainability & 2.67 & 1.13\\
      Scalability & 2.27 & 0.40\\
      Performance & 2.20 & 0.07\\
      \hline
    \end{tabular}
  \end{center}
\end{table}
      
%%%%%%%%%%%%%%%%%%%%%%%%%%%%%%%%%%%%
\section{Ways to improve maintainability (Q5 \& Q6)}

%%%%%%%%%%%%%%%%%%%%%%%%%%%%%%%%%%%%
\section{Experience vs. prioritization of maintainability}

%%%%%%%%%%%%%%%%%%%%%%%%%%%%%%%%%%%%
\section{Experience vs. suggested ways to improve maintainability}


%%%%%%%%%%%%%%%%%%%%%%%%%%%%%%%%%%%%%%%%%%%%%%%%%%%%%%%%%%%%%%%%%%%%%%%%%%%%%%
\chapter{Discussion}


%%%%%%%%%%%%%%%%%%%%%%%%%%%%%%%%%%%%%%%%%%%%%%%%%%%%%%%%%%%%%%%%%%%%%%%%%%%%%
\chapter{Conclusions}


%%%%%%%%%%%%%%%%%%%%%%%%%%%%%%%%%%%%%%%%%%%%%%%%%%%%%%%%%%%%%%%%%%%%%%%%%%%%%%
\printbibliography

\nocite{*}

%%%%%%%%%%%%%%%%%%%%%%%%%%%%%%%%%%%%%%%%%%%%%%%%%%%%%%%%%%%%%%%%%%%%%%%%%%%%%%
\appendix


\end{document}
