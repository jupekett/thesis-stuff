\documentclass[utf8,english]{gradu3}

% NOTE: This must be the last \usepackage in the whole document!
\usepackage[bookmarksopen,bookmarksnumbered,linktocpage]{hyperref}

\addbibresource{research-plan.bib} % The file name of your bibliography database

\begin{document}

\title{Maintainability in cloud-native architecture}
\type{Research plan} % not needed in the actual thesis
\translatedtitle{Ylläpidettävyys pilvinatiiveissa arkkitehtuureissa}
\studyline{Software and Telecommunication Technology}
\avainsanat{%
  ylläpito,
  ylläpidettävyys
  julkinen pilvi,
  pilvinatiivi,
  pilviarkkitehti,
  arkkitehtuuri,
  ohjelmistoarkkitehtuuri,
  pro gradu -tutkielmat
}
\keywords{%
  maintenance,
  maintainability,
  public cloud,
  cloud-native,
  cloud architect,
  architecture,
  software architecture,
  Master's Theses}
\tiivistelma{%
  Suomenkielinen tiivistelmä tulee tähän.
}
\abstract{%
  English abstract goes here.
}

\author{Juho Kettunen}
\contactinformation{\texttt{juho.kettunen@student.jyu.fi}}
\supervisor{Oleksiy Khriyenko}

\maketitle

\mainmatter

%%%%%%%%%%%%%%%%%%%%%%%%%%%%%%%%%%%%%%%%%%%%%%%%%%%%%%%%%%%%%%%%%%%%%%%%%%%%%%
\chapter{Introduction}

Goal of this thesis is to find out how maintainability is addressed during the
architectural design phase of cloud-native software development lifecycle.
I chose this topic for three reasons:
\begin{enumerate}
  \item Choices made during the architectural design phase cascade into
        development and maintenance phases. Mistakes and oversights are
        slower and more expensive to correct later.
  \item Maintenance phase of the software lifecycle is prevalent.
        It takes up the majority of total lifetime and costs.
  \item Cloud-native approach helps reduce time-to-market and move costs from
        capital expenditure to operational expenditure. Utilizing a public cloud
        platform allows you to leverage a highly scalable, reliable and secure
        infrastructure and a wide variety of easily integratable services.
\end{enumerate}


%%%%%%%%%%%%%%%%%%%%%%%%%%%%%%%%%%%%%%%%%%%%%%%%%%%%%%%%%%%%%%%%%%%%%%%%%%%%%%
\chapter{Literature review}
I will use these sources for literature:
\begin{itemize}
  \item \textcite{jykdok}
  \item \textcite{google-scholar}
  \item Snowballing from bibliographies of promising sources, e.g. \textcite{thesis-time-tracking}
  \item + other databases or seach engines that are discovered during research
\end{itemize}

Example search terms:
\begin{itemize}
  \item "cloud native" AND maintainability
  \item "cloud native" AND architecture
  \item architecture AND maintainability
  \item "cloud native" AND architecture AND maintainability
\end{itemize}


%%%%%%%%%%%%%%%%%%%%%%%%%%%%%%%%%%%%%%%%%%%%%%%%%%%%%%%%%%%%%%%%%%%%%%%%%%%%%%
\chapter{Research questions}

I aim to answer three research questions:
\begin{itemize}
  \item How much importance do cloud architects place on maintainability during
        the design phase?
  \item How to address maintainability concerns when architecting cloud-native
        applications?
  \item Do the recommendations from literature match the views of architects
        working in the field?
\end{itemize}


%%%%%%%%%%%%%%%%%%%%%%%%%%%%%%%%%%%%%%%%%%%%%%%%%%%%%%%%%%%%%%%%%%%%%%%%%%%%%%
\chapter{Method}

I will conduct a survey that offers real-life insights into the research questions.
It is targeted at cloud architects of company Nordcloud, where I work.

First I will ascertain the perceived importance of maintainability.
Then I will categorize the suggested approaches for addressing maintainability
and contrast this to the respondents' years of experience in the target domain.
Finally, I will compare the survey results to approaches suggested in the literature.


%%%%%%%%%%%%%%%%%%%%%%%%%%%%%%%%%%%%%%%%%%%%%%%%%%%%%%%%%%%%%%%%%%%%%%%%%%%%%%
\chapter{Gathering the data}

\section{What}
Survey will manifest as a Google Form.
It will include a data collection disclaimer and an agreement checkbox to proceed.

Survey might include these questions:
\begin{enumerate}
  \item How would you prioritize these software quality metrics when designing cloud-native
        architecture?
  \item How can one address maintainability through platform- and technology choices?
  \item How can one address maintainability through architecture?
  \item Years of experience with cloud-native architecture?
  \item Years of experience with software architecture in general?
  \item Years of experience in IT?
\end{enumerate}

Question 1 will have a selection of software quality metrics that the architect should set in a priority order.

Questions 2 and 3 are free-text answers.
I will interpret the answers to find common categories.

Questions 4 through 6 are single-selection questions, with these ranges:
\begin{itemize}
  \item 0-2
  \item 2-5
  \item 5-10
  \item 10+
\end{itemize}

There will be an optional feedback question in the end of the survey.

\section{Who}

Target audience is cloud architects working at Nordcloud, an IBM company.

\section{How}
I will post links to the survey on Nordcloud's internal Slack channels to reach
most of our cloud architects.
These communities will likely be targeted:
\begin{itemize}
  \item \#tech-infra
  \item \#tech-dev
  \item \#aws
  \item \#azure
  \item \#google-cloud
\end{itemize}

Some of these communities have overlapping audience, but the total reach is about 1600 people.
I assume our company to employ a few dozen cloud architects.
My goal is to get 10-20 answers in order to conduct any statistical analysis on
the data.

\section{When}
Survey will be sent out during Q2 of 2023.
I will repost the survey in Slack up to 4 times once a week until I decide I
have enough answers to proceed.

\section{Storage of data and privacy concerns}
I will use my JYU Google account for all Google-related functions.

Survey will be anonymous.
The form requires logging in with a Google account, but email addresses are not collected.
Answers are stored in Google Forms until the thesis is complete,
after which I will delete the survey.

Analysis is stored in Google Sheets until the end of my study rights and
subsequent removal of the Google account.

Possible Python scripts used for analysis will be stored in my personal Github repository
perpetually.
This is OK because it won't contain any personal data.


%%%%%%%%%%%%%%%%%%%%%%%%%%%%%%%%%%%%%%%%%%%%%%%%%%%%%%%%%%%%%%%%%%%%%%%%%%%%%%
\chapter{Analysis}

\section{What and how}
\begin{itemize}
  \item Interpret and list categories from answers to questions 2 and 3.
  \item Calculate the most popular categories and point out possible outliers.
  \item Compare categories to solutions proposed in literature.
  \item Calculate statistical correlation between years of experience and:
        \begin{itemize}
          \item perceived importance of maintainability
          \item number of categories proposed in 2 and 3
          \item popularity of categories proposed in 2 and 3
        \end{itemize}
  \item (Create nice graphs to visualize the above)
\end{itemize}

\section{Tech used}
\begin{itemize}
  \item Google Forms
  \item Google Sheets
  \item Python scripting for data analysis and graphs, if needed
\end{itemize}


%%%%%%%%%%%%%%%%%%%%%%%%%%%%%%%%%%%%%%%%%%%%%%%%%%%%%%%%%%%%%%%%%%%%%%%%%%%%%%
\printbibliography


\end{document}
